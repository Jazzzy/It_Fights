
En un contexto académico existe un campo en expansión dedicado a investigar sobre inteligencia artificial en juegos humanos. Dentro de la industria de los videojuegos existen puestos tales como \textit{programador de IA} que trabajan en desarrollar un comportamiento adecuado para personajes u otros aspectos de sus productos. Además existen también grupos que se auto-identifican como \textit{expertos en IA de videojuegos}.

\bigskip

Todos estos grupos tienen su propio campo con eventos y recursos separados:

\begin{itemize}
	\item La parte \textbf{académica} cuenta con las conferencias \textit{AIIDE}\footnote{\url{https://aaai.org/Conferences/AIIDE/aiide.php}} y \textit{CIG}\footnote{\url{http://www.cig2017.com/}}.
	\item La \textbf{industria} dedica eventos como la \textit{Game AI Conference}\footnote{\url{http://gameaiconf.com/}} y el \textit{GDC AI Summit}\footnote{\url{http://www.gdconf.com/conference/ai.html}} a enseñar sobre estos temas. Cuentan con una web dedicada unicamente a expandir el conocimiento existente llamada \textit{AIGameDev}\footnote{\url{http://aigamedev.com/}}.
\end{itemize}

\bigskip

Sin embargo, es sorprendente la ínfima colaboración existente entre las dos partes, de hecho, algunos componentes de ambas partes afirman que existe ignorancia y cierta desconfianza sobre el trabajo de la otra. Pese a que existan algunas excepciones a la regla esto no significa que no se esté desaprovechando un campo de trabajo que podría ser de utilidad a ambas partes\cite{blog:emerging}.

\bigskip

Pese a que existen trenes de pensamiento que afirman lo beneficioso que podría ser una colaboración\cite{cio}, hay una serie de razones por las cuales no se aúnan esfuerzos ni se intenta comprender a la otra parte, algunas de las cuales son:

\begin{itemize}
	\item \textbf{Se considera el éxito de forma distinta}: El contexto académico más puro premia las publicaciones y citas, no da importancia a convertir el trabajo realizado o prototipo en un producto listo para un mercado como el de los videojuegos. La industria se preocupa de realizar un producto que genere dinero y no tanto de las estrategias utilizadas para ello. Intentar algo que puede no funcionar se considera un riesgo innecesario.
	\item \textbf{Desconfianza mutua}: Una pequeña pero ruidosa parte de ambos mundos desconfía de las capacidades de los otros. En un caso porque se considera la falta de conocimientos aplicados y en el otro porque se piensa que la otra parte no tiene experiencia creando un producto final para el público.
	\item \textbf{La parte académica tiene dificultades para entrar}: Hay múltiples razones que hacen que investigadores decidan no colaborar con la industria, algunas de las cuales pueden ser:
		\begin{itemize}
			\item No están familiarizados con el contexto de desarrollo de un videojuego.
			\item No existen productos abiertos sobre los que empezar a trabajar a causa de intentar proteger la propiedad intelectual.
			\item Se prioriza un tipo de juegos equivocado. Generalmente los videojuegos que permiten implementar IA no tienen requisitos complejos con respecto a la misma.
		\end{itemize}
	\item \textbf{Los problemas son diferentes}: Muchos desarrolladores consideran los problemas académicos muy abstractos mientras que los investigadores ven los problemas de la industria como muy específicos o poco complejos.
\end{itemize}

\bigskip

Ejemplos como el producto que nos atañe pueden, de forma indirecta, mostrar los beneficios de la unión de ambos mundos. Se debe considerar que el trabajador no tiene experiencia aplicada en ninguno de los dos mundos, industria o investigación, dada su condición de estudiante. Sin embargo, si alguien con más interés que conocimientos puede hacer funcionar una pequeña demostración en relativamente poco tiempo no hay razón para pensar que unir esfuerzos a una mayor escala no sería beneficioso.
