\chapter{Introducción}

En un contexto académico existe un campo en expansión dedicado a investigar sobre inteligencia artificial en juegos humanos. Dentro de la industria de los videojuegos existen puestos tales como \textit{programador de IA} que trabajan en desarrollar un comportamiento adecuado para personajes u otros aspectos de sus productos. Además existen también grupos que se auto-identifican como \textit{expertos en IA de videojuegos}.

\bigskip

Todos estos grupos tienen su propio campo con eventos y recursos separados:

\begin{itemize}
	\item La parte \textbf{académica} cuenta con las conferencias \textit{AIIDE}\footnote{\url{https://aaai.org/Conferences/AIIDE/aiide.php}} y \textit{CIG}\footnote{\url{http://www.cig2017.com/}}.
	\item La \textbf{industria} dedica eventos como la \textit{Game AI Conference}\footnote{\url{http://gameaiconf.com/}} y el \textit{GDC AI Summit}\footnote{\url{http://www.gdconf.com/conference/ai.html}} a enseñar sobre estos temas. Cuentan con una web dedicada unicamente a expandir el conocimiento existente llamada \textit{AIGameDev}\footnote{\url{http://aigamedev.com/}}.
\end{itemize}

\bigskip

Sin embargo, es sorprendente la ínfima colaboración existente entre las dos partes, de hecho, algunos componentes de ambas partes afirman que existe ignorancia y cierta desconfianza sobre el trabajo de la otra. Pese a que existan algunas excepciones a la regla esto no significa que no se esté desaprovechando un campo de trabajo que podría ser de utilidad a ambas partes\cite{blog:emerging}.

\bigskip

Pese a que existen trenes de pensamiento que afirman lo beneficioso que podría ser una colaboración\cite{cio}, hay una serie de razones por las cuales no se aúnan esfuerzos ni se intenta comprender a la otra parte, algunas de las cuales son:

\begin{itemize}
	\item \textbf{Se considera el éxito de forma distinta}: El contexto académico más puro premia las publicaciones y citas, no da importancia a convertir el trabajo realizado o prototipo en un producto listo para un mercado como el de los videojuegos. La industria se preocupa de realizar un producto que genere dinero y no tanto de las estrategias utilizadas para ello. Intentar algo que puede no funcionar se considera un riesgo innecesario.
	\item \textbf{Desconfianza mutua}: Una pequeña pero ruidosa parte de ambos mundos desconfía de las capacidades de los otros. En un caso porque se considera la falta de conocimientos aplicados y en el otro porque se piensa que la otra parte no tiene experiencia creando un producto final para el público.
	\item \textbf{La parte académica tiene dificultades para entrar}: Hay múltiples razones que hacen que investigadores decidan no colaborar con la industria, algunas de las cuales pueden ser:
		\begin{itemize}
			\item No están familiarizados con el contexto de desarrollo de un videojuego.
			\item No existen productos abiertos sobre los que empezar a trabajar a causa de intentar proteger la propiedad intelectual.
			\item Se prioriza un tipo de juegos equivocado. Generalmente los videojuegos que permiten implementar IA no tienen requisitos complejos con respecto a la misma.
		\end{itemize}
	\item \textbf{Los problemas son diferentes}: Muchos desarrolladores consideran los problemas académicos muy abstractos mientras que los investigadores ven los problemas de la industria como muy específicos o poco complejos.
\end{itemize}

\bigskip

Ejemplos como el producto que nos atañe pueden, de forma indirecta, mostrar los beneficios de la unión de ambos mundos. Se debe considerar que el trabajador no tiene experiencia aplicada en ninguno de los dos mundos, industria o investigación, dada su condición de estudiante. Sin embargo, si alguien con más interés que conocimientos puede hacer funcionar una pequeña demostración en relativamente poco tiempo no hay razón para pensar que unir esfuerzos a una mayor escala no sería beneficioso.

\section{Videojuego y Agente}

En este apartado es donde se explica el tipo de videojuego a implementar. Además, se deberá relacionar dicho videojuego con las técnicas aplicadas.

\bigskip

La materialización de uno de los riesgos del proyecto descrito en el apartado \ref{AC} ha tenido un impacto relevante sobre la complejidad y completitud del agente. La primera aproximación al trabajo implicaba utilizar una demostración de la aplicación del videojuego realizada en un motor conocido pero surgió la necesidad de implementar la aplicación desde cero.

\bigskip

Esto se relaciona con los problemas mencionados en la sección anterior. Y es que no existe una suficiente variedad de herramientas para trabajar con técnicas de inteligencia artificial en videojuegos, solo un grupo pequeño de competiciones y videojuegos ya sobre-explotados.

\bigskip

Como solución al problema se ha implementado una aplicación sin variar las mecánicas previamente definidas. Se pasa ahora a explicar el funcionamiento del tipo de videojuego que se ha hecho para los lectores no familiarizados.

\subsection{Definición del videojuego}

En su definición más general, la aplicación se define como un videojuego de lucha uno contra uno \textit{Top-Down} en dos dimensiones. Esto quiere decir lo siguiente:

\begin{figure}
	\centerline{\includegraphics[width=6cm]{otros/manual/angulos.png}}
	\caption{Ángulos posibles}
	\label{mec:angulos}
\end{figure}

\begin{itemize}
	\item \textbf{Juego de lucha uno contra uno o \textit{1v1}}: Implica que dos personajes con exactamente las mismas capacidades, acciones posibles y atributos competirán en el mismo entorno. En lo que a la competición se refiere no existe ninguna entidad más involucrada por lo que el factor decisivo a la hora de determinar un ganador serán las habilidades de quien sea que controle a los personajes.
	\item \textbf{Top-Down}: Vista que en videojuegos se puede referir tanto a un plano cenital\footnote{en fotografía, el punto de vista de la cámara se encuentra perpendicular al suelo} como a un plano picado\footnote{en fotografía, el punto de vista de la cámara tiene un ángulo superior a 45 grados pero sin llegar a 90 con respecto al suelo}, se pueden ver ejemplos en la figura \ref{mec:angulos}. En el presente caso se utiliza un \textbf{plano picado}.
	\item \textbf{En dos dimensiones}: Pues el contenido del videojuego son imágenes planas dibujadas para representar los diferentes estados de los personajes, en ningún momento se utiliza ningún modelo entres dimensiones.
\end{itemize}

\bigskip

En lo referente a las \textbf{mecánicas} presentes en el combate se ha optado por un subconjunto de acciones reducido pero con cierta profundidad. Un personaje puede moverse libremente por la zona de combate que se limita por cuatro paredes (habitación rectangular). Además podrá atacar en la zona justo enfrente de él y defenderse de ataques que vienen desde cualquier dirección. Se explican las interacciones entre mecánicas en las siguientes líneas:

\begin{itemize}
	\item \textbf{Movimiento}: El hecho de moverse libremente en una área limitada hace que se genere naturalmente la necesidad de evitar situaciones en las que no se pueda escapar del contrincante. De forma similar al \textit{control del ring} en boxeo, es necesario tener hacia donde moverse si se requiere por lo que se añade la complejidad de evitar quedar atrapado entre los muros y el enemigo. Además la única forma de cambiar el ángulo hacia el que se está mirando es moverse hacia esa dirección.
	\item \textbf{Ataque}: Al poder atacar simplemente en la zona directamente enfrente del personaje se hace relevante la dirección hacia la que el mismo está mirando. Pese a que pueda parecer baladí esto hace que se beneficie una actitud agresiva al tener que moverte hacia la dirección del enemigo justo antes de atacar para estar mirando en su dirección.
	\item \textbf{Defensa}: Existe la posibilidad de ejecutar un movimiento defensivo que protege del ataque. Es la maniobra que representa un alto riesgo pero una alta recompensa ya que si se consigue defenderse de un ataque se proporciona la posibilidad de realizar un ataque propio sin consecuencias. Por otro lado, si la defensa no recibe ataques se entra en un periodo de descanso o \textit{cooldown} en el que el personaje es vulnerable.
\end{itemize}

\bigskip

La interacción entre movimiento, defensa y ataque crea situaciones en las que no está determinada la mejor estrategia claramente. Un estilo agresivo es castigado por un estilo defensivo, el estilo defensivo pierde ante un jugador pasivo que simplemente espera a que se entre en la fase vulnerable de la defensa y a su vez el jugador pasivo perderá ante uno agresivo. La clásica fórmula de \textit{piedra-papel-tijeras} aplicada de un contexto continuo ha probado ser efectiva en diseño de videojuegos desde el nacimiento de la industria.

\bigskip

Al no favorecer ningún estilo de juego concreto se abren posibilidades sobre que comportamiento es el idóneo para un posible agente complejo. En el siguiente apartado se comenta el algoritmo utilizado para aprender y jugar al videojuego.

\section{Prototipo de Unity}

Dado que el prototipo inicial realizado en Unity ha tenido un impacto significativo en el proyecto, tal y como se muestra en la sección \ref{AC}, se dedica este subapartado a mostrar los problemas que ha generado el elegir esta tecnología como primera opción.

\bigskip

En las figuras \ref{unity:combate} y \ref{unity:limite} podemos observar dos capturas del prototipo. En ambos casos se observan las vallas pensadas para limitar el área de combate. Las mismas cuentan con atributos que hacen que generen colisiones, es decir, que los personajes no puedan atravesarlas en ningún caso.

\bigskip

Sin embargo, al escalar el tiempo por valores relativamente pequeños, este tipo de funcionalidades dejan de funcionar ya que el motor tarda demasiado en componer cada uno de los fotogramas, haciendo que en muchas ocasiones en un fotograma se calcule que el personaje está en un lado de la valla y en el siguiente esté en el otro sin haber entrado en contacto con la misma.

\bigskip

Este tipo de fallos son inaceptables ya que rompen el funcionamiento de las mecánicas de juego al realizar las simulaciones, haciendo que un posible agente no pudiera aprender de forma eficaz y eficiente.

\bigskip

Algo parecido sucede con los ataques. Al intentar acelerar el tiempo y realizar un ataque ocurre con cierta frecuencia que el motor se salta los fotogramas en los que realmente se está comprobando si se hace daño al enemigo, inutilizando esta mecánica también.

\bigskip

En aras de permitir una visualización más comprensible de los problemas se pone a disposición un vídeo en \textit{YouTube}\footnote{disponible en: \url{https://youtu.be/PY4H8dk8zcU}} en el que se muestra concretamente el problema de la valla. Es cierto que en este caso sí se está mostrando la aplicación y no se está eliminando la visualización para aumentar el rendimiento, sin embargo se ha comprobado que los problemas son los mismos incluso sin visualización. Además el tiempo solo está escalado 20 veces, lo que se esperaría que fuera soportable para el motor pero no es así. Como comparación, en el producto final se escala unas 600 veces sin problema alguno.

\begin{figure}
	\centerline{\includegraphics[width=15cm]{otros/otrasCapturas/valla1.png}}
	\caption{Captura del prototipo en la zona de combate}
	\label{unity:combate}
\end{figure}

\begin{figure}
	\centerline{\includegraphics[width=15cm]{otros/otrasCapturas/valla2.png}}
	\caption{Captura del prototipo en el límite del mapa}
	\label{unity:limite}
\end{figure}


