\markboth{BIBLIOGRAFÍA}{BIBLIOGRAFÍA}
\addcontentsline{toc}{chapter}{Bibliografía}

\todo {Completar bibliografía}

\begin{thebibliography}{99}
	
\bibitem{definiendo_arquitectura} Definición de arquitectura según la ISO/IEC/IEEE 42010:2011. Artículo referenciado ({\it http://www.iso-architecture.org/ieee-1471/defining-architecture.html}). Consultado el 2 de junio de 2017.

\bibitem{modelado_referencia}Grady Booch, James Rumbaugh, and Ivar Jacobson. El lenguaje unificado de modelado. Manual
de referencia. Addison Wesley, 1999.

\bibitem{ieee}IEEE recommended practice for software requirements specifications. Technical report, 1998.
	
% ----------------EJEMPLOS--------------------	
% EXEMPLO DE DOCUMENTO DESCARGADO DA WEB
\bibitem{cuda} Nvidia CUDA programming guide. Versión 2.0, 2010. Dispoñible en {\it http://www.nvidia.com}.

% EXEMPLO DE PÁXINA DA WIKIPEDIA
\bibitem{cdma} Acceso múltiple por división de código. Artigo da wikipedia ({\it http://es.wikipedia.org}). Consultado o 2 de xaneiro do 2010.

% EXEMEPLO DE LIBRO
\bibitem{gonzalez} R.C. Gonzalez e R.E. Woods, {\it Digital image processing}, 3ª edición, Prentice Hall, New York, 2007.

% EXEMPLO DE ARTIGO DE REVISTA
\bibitem{patricia} P. González, J.C. Cartex e T.F. Pelas, ``Parallel computation of wavelet transforms using the lifting scheme'', {\it Journal of Supercomputing}, vol. 18, no. 4, pp. 141-152, junio 2001.
\end{thebibliography}

