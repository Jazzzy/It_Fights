\markboth{BIBLIOGRAFÍA}{BIBLIOGRAFÍA}
\addcontentsline{toc}{chapter}{Bibliografía}


\begin{thebibliography}{99}
	
\bibitem{cio} How video game AI is changing the world. Artículo de CIO ({\it \url{http://www.cio.com/article/3160106/artificial-intelligence/how-video-game-ai-is-changing-the-world.html}}).

\bibitem{modern} Peter Norvig, Stuart J. Russell. Artificial Intelligence: A Modern Approach, Tercera Edición. Prentice Hall (2009)

\bibitem{wiki:ia} Artificial intelligence (video games). Artículo de la wikipedia (\textit{\url{https://en.wikipedia.org/wiki/Artificial\_intelligence\_(video\_games)}}).
	
\bibitem{blog:emerging}Emergent Intelligence in Games. Publicación de Jon Radoff (archivado en \textit{\href{https://web.archive.org/web/20110219183426/http://radoff.com/blog/2011/02/17/emergent-intelligence-games/}{https://web.archive.org/web/20110219183426/http://radoff.com/blog/\\2011/02/17/emergent-intelligence-games/}}).
	
\bibitem{pmbok}PMI. Guía de los fundamentos para la dirección de proyectos (Guía del PMBOK), Quinta edición. PMI, 2013. 	
	
\bibitem{definiendo_arquitectura} Definición de arquitectura según la ISO/IEC/IEEE 42010:2011. Artículo referenciado ({\it \url{http://www.iso-architecture.org/ieee-1471/defining-architecture.html}}).

\bibitem{modelado_referencia}Grady Booch, James Rumbaugh, and Ivar Jacobson. El lenguaje unificado de modelado. Manual
de referencia. Addison Wesley, 1999.

\bibitem{ieee}IEEE recommended practice for software requirements specifications. Technical report, 1998.

\bibitem{nielsen}Nielsen, J., and Molich, R. Heuristic evaluation of user interfaces, Proc. ACM CHI'90 Conf. (Seattle, WA, 1-5 April), 249-256, 1990.

\bibitem{inteco}INTECO. Guía práctica de gestión de configuración LNCS. INTECO, 2008.

\bibitem{chaos}The Standish Group. Chaos Report. The Standish Group, 2016.

\bibitem{vitae}Vitae. Estudio salarial 2015-2016. Vitae Consultores, 2016.

\bibitem{game_engine}Jason Gregory. Game Engine Architecture, Second Edition 2nd. A. K. Peters, Ltd. Natick, MA, USA, 2014.

	
% ----------------EJEMPLOS--------------------	
% EXEMPLO DE DOCUMENTO DESCARGADO DA WEB
%\bibitem{cuda} Nvidia CUDA programming guide. Versión 2.0, 2010. Dispoñible en {\it http://www.nvidia.com}.

% EXEMPLO DE PÁXINA DA WIKIPEDIA
%\bibitem{cdma} Acceso múltiple por división de código. Artigo da wikipedia ({\it http://es.wikipedia.org}). Consultado o 2 de xaneiro do 2010.

% EXEMEPLO DE LIBRO
%\bibitem{gonzalez} R.C. Gonzalez e R.E. Woods, {\it Digital image processing}, 3ª edición, Prentice Hall, New York, 2007.

% EXEMPLO DE ARTIGO DE REVISTA
%\bibitem{patricia} P. González, J.C. Cartex e T.F. Pelas, ``Parallel computation of wavelet transforms using the lifting scheme'', {\it Journal of Supercomputing}, vol. 18, no. 4, pp. 141-152, junio 2001.
\end{thebibliography}
Todos los enlaces de la bibliografía han sido comprobados por última vez el \textbf{2 de julio de 2017}.
