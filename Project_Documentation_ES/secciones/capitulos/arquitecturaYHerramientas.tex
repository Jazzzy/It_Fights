\chapter{Arquitectura y herramientas}

El IEEE define arquitectura como los conceptos fundamentales o propiedades de un sistema en su entorno que se encarnan en sus elementos, relaciones y los principios de su diseño y evolución \cite{definiendo_arquitectura}.
Como referencia el propio IEEE, es el estándar ISO 12207 el que identifica un diseño arquitectónico y un diseño detallado y desgranado en lo que a un sistema se refiere. A partir de esto se obtiene que el diseño de la arquitectura de un sistema describe la estructura y la organización del mismo, es decir, se centra en los subsistemas o componentes que forman el sistema completo y las relaciones entre los mismos desde un punto de vista de alto nivel.

\bigskip

Contando ahora con una idea definida de cuales son los objetivos y requisitos de nuestro proyecto, nos centraremos en este capítulo en el diseño a alto nivel y en la arquitectura a la cual nos referimos en el párrafo anterior. Para ello utilizaremos representaciones gráficas a alto nivel del sistema y detallaremos los elementos arquitectónicos del producto y que tecnologías han sido usadas en los mismos así como las herramientas de las que hemos hecho uso para llevar a cabo el proyecto en su totalidad.

\section{Arquitectura del sistema}

\todo{Completar Arquitectura del sistema}

\section{Herramientas utilizadas}

\todo{Completar Herramientas utilizadas}