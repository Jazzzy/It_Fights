\chapter{Gestión del proyecto}

Dentro de la ingeniería de software, una de las partes esenciales para la realización de proyectos considerados exitosos es la gestión de proyectos. La realización de una buena gestión no se puede considerar, ni mucho menos, una garantía de que el proyecto a gestionar vaya a resultar un éxito. Sin embargo, si elegimos ignorar o realizar una mala gestión de nuestro proyecto sí podremos considerar que nos encontraremos en una situación proclive para un proyecto fallido. Durante toda la extensión del ciclo de vida de nuestro proyecto se utilizará la gestión de proyectos como un método que nos ayudará a lograr la obtención de un producto final ajustado a todas las necesidades y restricciones presentes, sea cual sea la índole de las mismas (tiempo, costes, requisitos, etc.).

\bigskip

Dedicaremos este capítulo a definir y explicar la gestión de nuestro proyecto en todas sus partes. Se determinará y explicará el análisis de riesgos, la metodología de desarrollo empleada, la gestión de configuración, la planificación temporal y la estimación de costes.

\section{Gestión de riesgos}

Citando al PMBOK \cite{pmbok} la definición de riesgo es:

\begin{quote}
	\textit{
		“[...]un evento o condición incierta que, de producirse, tiene un efecto positivo o
		negativo en uno o más de los objetivos del proyecto, tales como el alcance, el cronograma, el costo y la calidad.”
	}
\end{quote}

Las causas de un riesgo pueden ser varias y diversas y, si este finalmente se diera, los impactos que puede producir también pueden ser numerosos. Dentro de las causas, su índole puede ser de tipos muy variados, desde un requisito mal especificado a una restricción que no existe pasando por supuestos que no se ajustan a realidad o fallos en procesos externos al proyecto. Si alguna de estas situaciones se produce podría haber un impacto relevante sobre los objetivos definidos previamente para el proyecto como lo serían el alcance, el coste, el cronograma o la calidad.

\bigskip

El impacto suele ser generalmente negativo aunque ocasionalmente puede ser beneficioso para el proyecto. En el proyecto al que se refiere esta memoria realizaremos una gestión de riesgos enfocada a los que tienen una naturaleza negativa y que por lo tanto podrían afectar a los objetivos de forma perjudicial. Una vez dicho esto, hay que considerar las diferentes estrategias que se pueden utilizar para abordar las amenazas o riesgos con impacto negativo en caso de materializarse y serán las siguientes:

\begin{itemize}
	
	\item \textbf{Evitar}: Siguiendo esta estrategia de respuesta el equipo del proyecto intentará eliminar la amenaza o proteger al proyecto del posible impacto de forma preventiva. Comúnmente implica modificar la planificación para evitar completamente la amenaza. 
	
	\item \textbf{Transferir}: Al transferir un riesgo el equipo traslada el impacto asociado al mismo a un tercero, librándose así de la responsabilidad de dar una respuesta si el riesgo se da. Suele incluir el pago de una prima de riesgo a la entidad que asumirá el impacto del mismo. 
	
	\item \textbf{Mitigar}: Mitigar un riesgo implica actuar para reducir o bien la probabilidad de un riesgo o bien el impacto que este tendría en el proyecto, llevando uno de estos aspectos o ambos a un umbral aceptable. Habitualmente es más efectivo realizar acciones preventivas que intentar lidiar con el riesgo una vez que ha ocurrido. 
	
	\item \textbf{Aceptar}: Si el equipo decide reconocer el riesgo y no hacer nada al respecto a menos que este se materialice la estrategia que se está utilizando es la de aceptar dicho riesgo. Esta aproximación se suele utilizar si no es posible o rentable utilizar alguna de las otras estrategias. Esto puede implicar no realizar ninguna acción si el riesgo ocurra (estrategia pasiva) o establecer una reserva para contingencias que pueda aportar los recursos, tiempo o dinero necesarios para gestionar el riesgo (estrategia activa).
	
\end{itemize}

Dada la índole del proyecto y considerando el hecho de que solo hay un trabajador encargado del mismo se intentará evitar la transferencia de un riesgo al ser esta estrategia propensa a generar costes adicionales. Por razones similares, la aceptación de un riesgo se deberá considerar única y exclusivamente si ninguna de las otras estrategias es aplicable.

Para realizar un buen análisis de riesgos se debe comenzar por una identificación de la mayor cantidad de riesgos posible dentro de unos límites lógicos para luego analizar la importancia de cada uno teniendo en cuenta su probabilidad e impacto. Finalmente se deberán establecer estrategias a seguir en caso de que los riesgos se den, priorizando los más importantes en caso de que no sea posible gestionarlos todos.

\subsection{Especificación de riesgos}

Se dedicará este apartado a mostrar las especificaciones formales de los riesgos que contendrán su descripción, probabilidad, impacto y los planes diseñados para cada uno de ellos. La probabilidad\footnote{indicada como $p$ dentro de la tabla}  e impacto\footnote{indicada como $i$ dentro de la tabla} podrán tomar los valores definidos en el Cuadro \ref{tab:probabilidad_impacto}. El identificador utilizado para los riesgos tendrá la forma RSK\footnote{En referencia a \textit{"risk"}, riesgo en inglés}-\textit{N}


\begin{table}
	\begin{center}
		\caption{Valores posibles de probabilidad e impacto}
		\label{tab:probabilidad_impacto}
	\begin{tabular}{ | c | c | c | } 
		\hline
		
		\textbf{Calificativo} &
		\textbf{Probabilidad}&
		\textbf{Impacto}\\
		
		\hline
		Muy Bajo 
		&
		$p<0,10$
		&
		$i<0,10$
		\\ 
		
		\hline
		Bajo 
		&
		$0,10<=p<0,30$
		&
		$0,10<=i<0,20$
		\\ 
		
		\hline
		Moderado 
		&
		$0,30<=p<0,50$
		&
		$0,20<=i<0,40$
		\\ 
		
		\hline
		Alto 
		&
		$0,50<=p<0,70$
		&
		$0,40<=i<0,80$
		\\ 
		
		\hline
		Muy Alto 
		&
		$0,70<=p$
		&
		$0,80<=i$
		\\ 
		
		
		\hline
	\end{tabular}
\end{center}
\end{table}

\newcounter{contador_riesgos}
\setcounter{contador_riesgos}{1}

\begin{center}
	\begin{tabular}{ | p{5.6cm} | p{8.5cm} | } 
		\hline
		
		\textbf{ID} & RSK-\arabic{contador_riesgos}
		\refstepcounter{contador_riesgos} \\
		
		\hline 
		\textbf{Nombre} &
		Retraso en la planificación\\ 
		
		\hline
		\textbf{Descripción} & 
		Se debe contemplar la posibilidad de que ocurran retrasos en la planificación dada la inexperiencia del desarrollador a la hora de trabajar en un proyecto de esta envergadura y con algunas de las tecnologías a utilizar.\\
		
		\hline 
		\textbf{Probabilidad de ocurrencia} &
		Alta\\
		
		\hline 
		\textbf{Impacto} &
		Muy Alto\\
		
		\hline 
		\textbf{Plan de Prevención} &
		\textbf{Mitigar}: Realizar reuniones periódicas con los tutores del proyecto para resolver dudas y ver el avance del mismo.\\
		
		\hline 
		\textbf{Plan de Contingencia} &
		\textbf{Aceptar}:Redefinir el alcance del proyecto volviendo a evaluar los requisitos para priorizar los mas necesarios dejando los no vitales fuera del proyecto.\\
		
	
		\hline
	\end{tabular}
\end{center}

\begin{center}
	\begin{tabular}{ | p{5.6cm} | p{8.5cm} | } 
		\hline
		
		\textbf{ID} & RSK-\arabic{contador_riesgos}
		\refstepcounter{contador_riesgos} \\
		
		\hline 
		\textbf{Nombre} &
		Cambio en los requisitos\\ 
		
		\hline
		\textbf{Descripción} & 
		Es posible que una vez avanzado el proyecto se considere que algunos de requisitos necesitan ser añadidos, modificados o eliminados, especialmente dado que se dedicará cierta parte del proyecto a investigar la utilidad de técnicas de inteligencia artificial sobre las cuales se desconoce si serán de utilidad.\\
		
		\hline 
		\textbf{Probabilidad de ocurrencia} &
		Alta\\
		
		\hline 
		\textbf{Impacto} &
		Alto\\
		
		\hline 
		\textbf{Plan de Prevención} &
		\textbf{Mitigar}: Realizar un diseño de la aplicación de la forma más modular posible para evitar grandes refactorizaciones a la hora de hacer modificaciones importantes.\\
		
		\hline 
		\textbf{Plan de Prevención 2} &
		\textbf{Mitigar}: Utilizar una metodología de trabajo ágil que sea capaz de aceptar y soportar cambios sustanciales en los requisitos del proyecto.\\
		
		
		\hline
	\end{tabular}
\end{center}

\begin{center}
	\begin{tabular}{ | p{5.6cm} | p{8.5cm} | } 
		\hline
		
		\textbf{ID} & RSK-\arabic{contador_riesgos}
		\refstepcounter{contador_riesgos} \\
		
		\hline 
		\textbf{Nombre} &
		Falta de comprensión de conceptos\\ 
		
		\hline
		\textbf{Descripción} & 
		Puede ocurrir que el desarrollador no esté tan familiarizado con los conceptos relacionados con inteligencia artificial como es necesario para la realización del proyecto, dificultando así la implementación correcta de cualquiera de sus técnicas.\\
		
		\hline 
		\textbf{Probabilidad de ocurrencia} &
		Media\\
		
		\hline 
		\textbf{Impacto} &
		Alto\\
		
		\hline 
		\textbf{Plan de Prevención} &
		\textbf{Mitigar}: Poner a disposición del desarrollador bibliografía que explique los conceptos que pueden ser necesarios.\\
		
		\hline 
		\textbf{Plan de Contingencia} &
		\textbf{Aceptar}: Dedicar las reuniones necesarias con los tutores a explicar los conceptos al desarrollador hasta que el nivel de comprensión de los mismos sea suficiente.\\
		
		
		\hline
	\end{tabular}
\end{center}

\begin{center}
	\begin{tabular}{ | p{5.6cm} | p{8.5cm} | } 
		\hline
		
		\textbf{ID} & RSK-\arabic{contador_riesgos}
		\refstepcounter{contador_riesgos} \\
		
		\hline 
		\textbf{Nombre} &
		Baja del desarrollador del proyecto\\ 
		
		\hline
		\textbf{Descripción} & 
		Se puede dar el caso de que, por diversas causas como problemas de salud, económicos o de otra índole, el desarrollador del proyecto se vea obligado a dejar de trabajar en él.\\
		
		\hline 
		\textbf{Probabilidad de ocurrencia} &
		Baja\\
		
		\hline 
		\textbf{Impacto} &
		Muy Alto\\
		
		
		\hline 
		\textbf{Plan de Contingencia} &
		\textbf{Aceptar}: Aplazar el proyecto hasta que el desarrollador se encuentre en condiciones de volver a trabajar en él.\\
		
		
		\hline
	\end{tabular}
\end{center}

\begin{center}
	\begin{tabular}{ | p{5.6cm} | p{8.5cm} | } 
		\hline
		
		\textbf{ID} & RSK-\arabic{contador_riesgos}
		\refstepcounter{contador_riesgos} \\
		
		\hline 
		\textbf{Nombre} &
		Baja de los tutores\\ 
		
		\hline
		\textbf{Descripción} & 
		Se puede dar el caso de que, por diversas causas como problemas de salud, económicos o de otra índole, uno o ambos tutores del proyecto se vean obligados a dejar de trabajar en él.\\
		
		\hline 
		\textbf{Probabilidad de ocurrencia} &
		Bajo\\
		
		\hline 
		\textbf{Impacto} &
		Alto\\
		
		
		\hline 
		\textbf{Plan de Contingencia} &
		\textbf{Aceptar}: En caso de que solo uno de los tutores deje de poder trabajar en el proyecto se continuará con el desarrollo con la ayuda del otro tutor. Si se da el caso de que ambos tienen que dejar de trabajar se deberá buscar un tutor o grupo de tutores como sustitutos.\\
		
		
		\hline
	\end{tabular}
\end{center}

\begin{center}
	\begin{tabular}{ | p{5.6cm} | p{8.5cm} | } 
		\hline
		
		\textbf{ID} & RSK-\arabic{contador_riesgos}
		\refstepcounter{contador_riesgos} \\
		
		\hline 
		\textbf{Nombre} &
		Fallo en el equipo de trabajo\\ 
		
		\hline
		\textbf{Descripción} & 
		Puede ocurrir que el equipo utilizado para el desarrollo quede inutilizado por razones de software o hardware lo que impide continuar con el proyecto en dicho sistema.\\
		
		\hline 
		\textbf{Probabilidad de ocurrencia} &
		Baja\\
		
		\hline 
		\textbf{Impacto} &
		Alto\\
		
		\hline 
		\textbf{Plan de Contingencia} &
		\textbf{Aceptar}: Se podrá hacer uso de equipos de la universidad para la realización del proyecto, si estos no son compatibles con alguna de las tecnologías usadas se necesitará adquirir un equipo sustituto cuya amortización tendrá que ser considerada dentro de los costes del proyecto.\\
		
		
		\hline
	\end{tabular}
\end{center}

\begin{center}
	\begin{tabular}{ | p{5.6cm} | p{8.5cm} | } 
		\hline
		
		\textbf{ID} & RSK-\arabic{contador_riesgos}
		\refstepcounter{contador_riesgos} \\
		
		\hline 
		\textbf{Nombre} &
		Perdida del proyecto y su documentación\\ 
		
		\hline
		\textbf{Descripción} & 
		Puede ocurrir que un fallo en el equipo de trabajo o en su almacenamiento haga inaccesible o elimine el topo o parte del trabajo realizado para el proyecto y/o para su documentación.\\
		
		\hline 
		\textbf{Probabilidad de ocurrencia} &
		Baja\\
		
		\hline 
		\textbf{Impacto} &
		Muy Alto\\
		
		\hline 
		\textbf{Plan de Prevención} &
		\textbf{Evitar}: Utilizar un sistema de control de versiones en la nube que permita guardar no solo todo el código y archivos necesarios para la ejecución del programa sino también todos los documentos que generan la documentación pudiendo recuperarlos desde cualquier otro equipo.\\
		
		
		\hline
	\end{tabular}
\end{center}

\begin{center}
	\begin{tabular}{ | p{5.6cm} | p{8.5cm} | } 
		\hline
		
		\textbf{ID} & RSK-\arabic{contador_riesgos}
		\refstepcounter{contador_riesgos} \\
		
		\hline 
		\textbf{Nombre} &
		Rendimiento insuficiente del videojuego\\ 
		
		\hline
		\textbf{Descripción} & 
		Puede ocurrir que, dada la inexperiencia como programador del desarrollador principal, el videojuego no cumpla los estándares de rendimiento esperados del mismo y no brinde una experiencia jugable.\\
		
		\hline 
		\textbf{Probabilidad de ocurrencia} &
		Baja\\
		
		\hline 
		\textbf{Impacto} &
		Muy Alto\\
		
		\hline 
		\textbf{Plan de Prevención} &
		\textbf{Mitigar}: Dedicar un tiempo en la fase de análisis y diseño a familiarizarse con las tecnologías y técnicas utilizadas en la programación de videojuegos cuando el rendimiento es un factor importante.\\
		
		\hline 
		\textbf{Plan de Contingencia} &
		\textbf{Aceptar}: Volver a realizar la etapa de diseño e implementación de forma que se eliminen las partes menos indispensables y más costosas del videojuego en aras de buscar un rendimiento superior.\\
		
		
		\hline
	\end{tabular}
\end{center}

\begin{center}
	\begin{tabular}{ | p{5.6cm} | p{8.5cm} | } 
		\hline
		
		\textbf{ID} & RSK-\arabic{contador_riesgos}
		\refstepcounter{contador_riesgos} \\
		
		\hline 
		\textbf{Nombre} &
		Imposibilidad de realizar simulaciones aceleradas\\ 
		
		\hline
		\textbf{Descripción} & 
		Dado que la realización de simulaciones dentro del entorno de un videojuego en tiempo real no es una práctica común es posible que las tecnologías escogidas no permitan realizar dicho proceso lo suficientemente rápido como para que el agente aprenda en un tiempo razonable.\\
		
		\hline 
		\textbf{Probabilidad de ocurrencia} &
		Moderada\\
		
		\hline 
		\textbf{Impacto} &
		Muy Alto\\
		
		\hline 
		\textbf{Plan de Prevención} &
		\textbf{Mitigar}: Realizar un prototipo inicial con la tecnología escogida para la implementación del videojuego lo más temprano posible en el proyecto con el fin de averiguar sus capacidades a la hora de acelerar las simulaciones. En este momento se considerará si es necesario cambiar la tecnología y/o realizar una implementación más eficiente.\\
		
		\hline 
		\textbf{Plan de Contingencia} &
		\textbf{Aceptar}: Cambiar la tecnología y/o realizar cambios en el diseño e implementación buscando un aumento en la eficiencia que posibilite realizar simulaciones a una velocidad suficiente.\\
		
		
		\hline
	\end{tabular}
\end{center}

\begin{center}
	\begin{tabular}{ | p{5.6cm} | p{8.5cm} | } 
		\hline
		
		\textbf{ID} & RSK-\arabic{contador_riesgos}
		\refstepcounter{contador_riesgos} \\
		
		\hline 
		\textbf{Nombre} &
		Agente sin capacidad de aprendizaje\\ 
		
		\hline
		\textbf{Descripción} & 
		Es posible que dada la inexperiencia del desarrollador en términos de implementación de técnicas de inteligencia artificial el funcionamiento no sea el correcto y el agente no sea capaz de aprender y actuar dentro del entorno del juego.\\
		
		\hline 
		\textbf{Probabilidad de ocurrencia} &
		Baja\\
		
		\hline 
		\textbf{Impacto} &
		Alto\\
		
		\hline 
		\textbf{Plan de Prevención} &
		\textbf{Mitigar}: Dedicar parte del tiempo en las reuniones con los tutores a comprobar el correcto funcionamiento de las técnicas implementadas.\\
		
		\hline 
		\textbf{Plan de Contingencia} &
		\textbf{Aceptar}: Cuando los errores en la implementación eviten el correcto funcionamiento del agente se dedicará una reunión específica a depurar dicha implementación y solucionar los errores presentes.\\
		
		
		\hline
	\end{tabular}
\end{center}

\subsection{Control de riesgos}

Controlar los riesgos implica implementar los planes de respuesta incluidos en la sección de especificación así como dar seguimiento a los riesgos identificados, monitorizarlos, identificar riesgos nuevos y en general evaluar el proceso de gestión de riesgos que se está realizando. A términos de este proyecto se considerará principalmente en el proceso de ejecutar las respuestas a los riesgos planificados que se han incluido en el apartado de especificación ya que, dado el tamaño y tiempo que se debe dedicar al proyecto, se podría correr el riesgo de dedicar demasiado tiempo a procesos de gestión de riesgos comparado con el tiempo dedicado al proyecto en si mismo.

\subsubsection{Acciones correctivas aplicadas}

Este apartado incluirá todo tipo de acción que se ha llevado a cabo para gestionar un riesgo que se ha materializado. En este sentido se especificarán las circunstancias en las que se ha dado el riesgo, la probabilidad e impacto una vez gestionado el riesgo y aplicado el plan apropiado, una descripción de las acciones realizadas y el efecto que las mismas han tenido en el proyecto.


\newcounter{contador_accion_correctiva}
\setcounter{contador_accion_correctiva}{1}

\begin{center}
	\begin{longtable}{ | p{5.6cm} | p{8.5cm} | } 
		
		\caption{Acción correctiva sobre la imposibilidad de realizar simulaciones aceleradas}\\
		
		\hline
		\textbf{ID} & AC-\arabic{contador_accion_correctiva}\label{tab:tabla_riesgo_escalado}
		\refstepcounter{contador_accion_correctiva} \\
		
		
		\hline 
		\textbf{ID del riesgo} & RSK-9: Imposibilidad de realizar simulaciones aceleradas \\
		
		\hline 
		\textbf{Nombre} &
		Agente sin capacidad de aprendizaje\\ 
		
		\hline
		\textbf{Descripción} &
		\pbox{8.4cm}{ 
			Al llevar a cabo el plan de prevención definido para este riesgo se descubrió que Unity, el motor elegido sobre el cual se desarrollaría la aplicación, no era capaz de escalar el tiempo de forma significativa sin que mecánicas básicas dentro del mismo dejaran de funcionar.\\\\El motor ofrece la funcionalidad de escalar el tiempo mediante el uso de la variable \textit{"Time.timeScale"} pero al intentar multiplicar la velocidad del tiempo interno del juego por valores relativamente pequeños como 20 o 30 las físicas del motor dejaban de funcionar correctamente. Para poner en contexto estos valores, en la implementación final realizada por el desarrollador el tiempo se escala por aproximadamente 600 veces sin que se modifique su funcionamiento en ningún sentido.
		}
		\\
		
		\hline
		\textbf{Acción correctiva} &
		\pbox{8.4cm}{ 
			La acción correctiva coincide con el plan de contingencia definido en la especificación del riesgo. Esto implicaba considerar una reimplementación de la aplicación ya fuera con la misma u otra tecnología. Dado que el origen del problema parecía estar relacionado con el motor utilizado y no con la implementación en si misma se decidió buscar tecnologías alternativas.\\\\Al considerar otros motores se observó que este tipo de funcionalidades era aún mas inusual de lo que en un principio parecía y había muchas posibilidades de que ninguno fuera capaz de cumplir nuestras necesidades.\\\\Como no se podía correr el riesgo de utilizar tiempo en preparar otro prototipo y que no funcionara otra vez se decidió hacer una implementación del videojuego con C++ y el uso de librerías gráficas de forma que fuera posible desactivar la visualización de los combates y así conseguir que todo el código que se ejecuta en las simulaciones fuera realizado por nosotros, dándonos un nivel de control superior y permitiendo realizar las optimizaciones pertinentes.
		}
		\\
		
		\hline
		\textbf{Efecto en el proyecto} &
		\pbox{8.4cm}{ 
			El impacto sobre el proyecto ha sido significativo pues la acción correctiva implica reconsiderar el alcance de algunos de los objetivos, principalmente los relacionados con la implementación de diversas técnicas de inteligencia artificial para comprobar su efectividad.\\\\Esto es debido a que el tiempo dedicado a implementar una aplicación de esta índole con un motor ampliamente usado por la comunidad y que aporta muchas facilidades es notablemente menor que el necesario para implementar en un lenguaje de programación como C++ un motor simple desde cero y agregar al mismo las funcionalidades necesarias.\\\\El plan de contingencia aplicado ha implicado no solo reconsiderar los requisitos hasta obtener los que se muestran en este documento, sino modificar la planificación temporal y agregar tareas relacionadas con el aprendizaje, diseño e implementación de la nueva aplicación como se muestra en los apartados correspondientes.
		}
		\\
		
		\hline 
		\textbf{Nueva probabilidad de \mbox{ocurrencia}} &
		Baja\\

		\hline 
		\textbf{Nuevo impacto} &
		Alto\\
		
		\hline
		

	\end{longtable}
\end{center}

\section{Gestión de la configuración}

\todo{Completar la gestión de la configuración}

\section{Metodología de desarrollo}

\todo{Completar la metodología}


\section{Planificación temporal}

\todo{Completar la planificación temporal}

\section{Análisis de costes}

\todo{Completar la estimación de costes}

\section{Plan de Gestión de las Comunicaciones}

\todo{Completar el plan de Gestión de las Comunicaciones}