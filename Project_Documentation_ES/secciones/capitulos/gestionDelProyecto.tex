\chapter{Gestión del proyecto}

Dentro de la ingeniería de software, una de las partes esenciales para la realización de proyectos considerados exitosos es la gestión de proyectos. La realización de una buena gestión no se puede considerar, ni mucho menos, una garantía de que el proyecto a gestionar vaya a resultar un éxito. Sin embargo, si elegimos ignorar o realizar una mala gestión de nuestro proyecto sí podremos considerar que nos encontraremos en una situación proclive para un proyecto fallido. Durante toda la extensión del ciclo de vida de nuestro proyecto se utilizará la gestión de proyectos como un método que nos ayudará a lograr la obtención de un producto final ajustado a todas las necesidades y restricciones presentes, sea cual sea la índole de las mismas (tiempo, costes, requisitos, etc.).

\bigskip

Dedicaremos este capítulo a definir y explicar la gestión de nuestro proyecto en todas sus partes. Se determinará y explicará el análisis de riesgos, la metodología de desarrollo empleada, la gestión de configuración, la planificación temporal y la estimación de costes.

\section{Análisis de riesgos}

\todo{Completar los riesgos necesarios}

\section{Metodología de desarrollo}

\todo{Completar la metodología}

\section{Gestión de la configuración}

\todo{Completar la gestión de la configuración}

\section{Planificación temporal}

\todo{Completar la planificación temporal}

\section{Estimación de costes}

\todo{Completar la estimación de costes}