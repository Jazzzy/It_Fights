\chapter{Prefacio}

El proyecto que nos atañe en este documento se enfoca en la creación de un videojuego con necesidades de comportamiento complejo por parte de un enemigo. Para lograr esto, los principales aspectos a abordar son la creación de la aplicación que permita realizar competiciones, su capacidad para permitir entrenar al agente y el agente en sí mismo que represente un competidor no solo apto pero también justo.

\bigskip

Se ha realizado una etapa de entrenamiento en la que el agente compite contra él mismo y otras implementaciones del enemigo con el fin de aprender los fundamentos del juego y tener el rendimiento suficiente para ganar frecuentemente a un jugador humano.

\section{Objetivos}

El objetivo global es implementar un videojuego que contenga a un agente capaz de controlar a uno de los personajes simulando un comportamiento competitivo. Más concretamente los sub-objetivos a abarcar son los siguientes:

\begin{enumerate}
	
	{\item {\bf Implementar el videojuego:}
		Se necesita una plataforma que permita tanto a un jugador humano como al agente interactuar con el entorno del videojuego siguiendo ambos las reglas que este mismo define.
	}

	{\item {\bf Implementar el agente con las técnicas escogidas:}
		Se requiere realizar la implementación del agente para simular un comportamiento competitivo.
	}

	{\item {\bf Realizar el entrenamiento del agente:}
		Se necesitará ejecutar un proceso de entrenamiento para que este adquiera la información necesaria para comportarse adecuadamente en el entorno competitivo del juego.
	}

	{\item {\bf Obtener datos sobre las capacidades del agente:}
		Se deberán obtener datos sobre el comportamiento del agente en el entorno del juego al competir con otras posibles implementaciones del mismo que no incluyan el uso de técnicas de inteligencia artificial.
	}

	{\item {\bf Analizar los resultados obtenidos:}
		Se recopilará la información obtenida durante las etapas del proyecto y se realizará un análisis que resuma lo que ha logrado el agente.
	}
	
\end{enumerate}

\section{Organización del documento}

La finalidad de este documento es presentar cómo se han resuelto los objetivos definidos para el proyecto. Para ello se explicarán las diferentes partes que forman el producto final así como las tareas que han sido realizadas a lo largo del proyecto y que han dado lugar al mismo tal y como se presenta.

\begin{itemize}
	\item En el \textbf{\textit{capítulo 2}} se introduce el contexto referente a videojuegos e inteligencia artificial. Se buscan presentar una serie de conceptos generales que permitan comprender fácilmente los capítulos posteriores.
	\item El \textbf{\textit{capítulo 3}} introduce al videojuego y a las implementaciones realizadas, así como el algoritmo utilizado, la otra implementación del agente y las técnicas de entrenamiento.
	\item El \textbf{\textit{capítulo 4}} contiene tanto el análisis de requisitos del proyecto como las diferentes partes que definen el alcance del mismo.
	\item El \textbf{\textit{capítulo 5}} documenta las etapas de gestión del proyecto que contienen: gestión de riesgos, gestión de la configuración, metodología empleada, planificación temporal, análisis de costes y plan de comunicaciones.
	\item En el \textbf{\textit{capítulo 6}} se introducen las partes de la aplicación especificando su arquitectura así como las herramientas utilizadas para su creación.
	\item Es en el \textbf{\textit{capítulo 7}} donde se puede ver la documentación asociada al diseño de implementación del producto final con los diagramas asociados.
	\item El \textbf{\textit{capítulo 8}} contiene las pruebas que verifican y validan el sistema.
	\item En el \textbf{\textit{capítulo 9}} se recogen las conclusiones obtenidas una vez finalizado el proyecto así como las posibles ampliaciones futuras que serían de utilidad.
	\item Finalmente, se agregan dos \textbf{\textit{apéndices}} que contienen el manual técnico y manual de usuario.
\end{itemize}



