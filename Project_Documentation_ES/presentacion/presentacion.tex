 
% Para una visualizacion correcta, generar el PDF
% Ni el DVI ni el PS se visualizan bien

% Elegir el estilo que se desee, hay cientos en la red

\documentclass{beamer}
\usepackage{beamerthemeshadow}
\usepackage{pgfpages}
\usepackage[galician]{babel}
%\usepackage[spanish]{babel}
\usepackage[utf8]{inputenc}
%\usepackage[latin1]{inputenc}

\setbeameroption{show notes on second screen}

\begin{document}
\title{Intelixencia Artificial aplicada a Videoxogos Top-Down en tempo real}
\subtitle{Grao en Enxeñaría Informática \\
Universidade de Santiago de Compostela}  
\author{Autor: Rubén Osorio López}
\institute{Titor: Manuel Mucientes Molina\\Cotitor: Pablo Rodrígez Mier}
\date{21 de xullo de 2017} 

\begin{frame}
\titlepage
\note{Esto é a defensa da memoria do traballo de fin de grado nombrado \textbf{Intelixencia Artificial aplicada a Videoxogos Top-Down en tempo real}, eu son o autor, Rubén Osorio López, e os tutores son Manuel Mucientes Molina e Pablo Rodrígez Mier}
\end{frame}

\begin{frame}
\frametitle{Táboa de contidos}\tableofcontents
\note{Durante esta presentación seguiremos unha estructura similar á memoria, centrándonos máis en alguns aspectos concretos do proxecto que expliquen en que consistiu o traballo realizado.}
\end{frame} 



\section{Introdución} 

\begin{frame}
\frametitle{Introdución} 


	\begin{itemize}
		\item Proxecto que aborda a creación dun \textbf{videoxogo} con necesidades de \textbf{comportamento complexo} por parte do inimigo.
	\end{itemize} 

\begin{block}{Videoxogo}
	\textbf{Loita 1 contra 1}, \textbf{Top-Down} en \textbf{dúas dimensións}
\end{block}

\begin{block}{Axente}
Capaz de percibir e actuar sobre o \textbf{entorno competitivo} do videoxogo mediante \textbf{sensores} e \textbf{actuadores}
\end{block}
\note{
	
	De forma xeral, búscase a creación dun videoxogo que requira un inimigo con comportamento complexo. O axente que representará o inimigo necesita ser un competidor capaz, para o que se realizou unha etapa de entrenamento na que optivo a información que necesitaba.\\
	
	\textbf{Loita 1 contra 1} significa que soamente dous perxonaxes competirán entre eles contando ambos coas mesmas capacidades, accións posibles e atributos. \textbf{Top-Down} refírese ó plano picado utilizado para visualizar o combate. Por outra parte que sea en \textbf{dúas dimensións} implica que todo o contido do videoxogo son imaxen planas debuxadas unha a unha, sen que existan modelos en tres dimensións.\\
	
	\textbf{Un axente} é aquilo capaz de percibir o entorno mediante \textbf{sensores} e actuar sobre o mesmo en consecuencia mediante \textbf{actuadores}, ambos son proporcionados pola súa interface co videoxogo. Ademáis atoparase nun entorno competitivo o que implica que buscará maximizar o seu \textbf{rendemento} mentres se minimiza o do contrincante.
	


}
\end{frame}




\subsection{Obxectivos}
\begin{frame}
\frametitle{Obxectivos} 
\begin{itemize}
	\pause
	\item Implementación do videoxogo \pause
	\item Implementación do axente \pause
	\item Realizar o entrenamento do axente \pause
	\item Obter datos sobre as capacidades do axente \pause
	\item Analizar os resultados obtidos \pause
\end{itemize}
\note{
	\begin{itemize}
		\item Implementación do videoxogo: Xa que necesitaremos unha plataforma que nos permita que o axente e o xogador interactúen seguindo unha serie de regras comúns para competir entre eles.
		\item Implementación do axente: Necesitarase implementar o axente capaz de desenvolverse correctamente durante a competición.
		\item Realizar o entrenamento do axente: O axente necesitará obter a información necesaria para logo comportarse adecuadamente gracias ó aprendido durante a etapa de entrenamento.
		\item Obter datos sobre as capacidades do axente: Obteránse datos sobre o rendemento do axente contra outras implementacións máis sinxelas.
		\item Analizar os resultados obtidos: Coa información obtida durante todo o proxecto, e especialmente na etapa anterior, realizarase un análise que describa o que conseguiu o axente.
\end{itemize}
}
\end{frame}


\section{Videoxogo basado en axentes}

\subsection{Mecánicas}
\begin{frame}
\frametitle{Mecánicas}

\begin{block}{Movemento}
Movemento libre nunha habitación rectangular.
\end{block}

\begin{block}{Ataque}
Permítese atacar a zona que se atopa cada donde o perxonaxe está mirando.
\end{block}

\begin{block}{Defensa}
Posibilidade de defenderse dun ataque permitindo atacar se a defensa ten éxito
\end{block}

\note{
\textbf{Movemento} libre nunha habitación rectangular que suma a complexidade de evitar situacións nas que non se poida escapar do contrincante por estar ó lado dunha parede ou unha esquina. Ademáis a única maneira de mirar cara unha dirección é mirar cara ela.\\
\textbf{Como} solo se permite atacar a zona directamente enfrente do personaxe é importante ter en conta cada donde se está mirando. Esto favorece unha actitude agresiva pois hai que moverse na dirección do enemigo antes de atacalo.\\
\textbf{Pódese} realizar unha maniobra defensiva de alto risco e alta recompensa que permite evitar un ataque. Se se evita con éxito poderase realizar un ataque propio pero se non serase vulnerable durante uns instantes.\\
\textbf{Esto fai que non exista unha estratexia idónea} pois un estilo agresivo perde contra un defensivo que á sua vez perde contra xogadores que busquen a contra do movemento defensivo, este último ademáis perde contra o xogador agresivo. Esta fórmula de \textbf{pedra, papel, tixeiras} demostrou ser ampliamente utilizada en diseño de videoxogos.\\
}
\end{frame}


\subsection{Prototipo de Unity}
\begin{frame}
\frametitle{Prototipo de Unity}
Primera implementación realizada con \textbf{Unity3D}, estándar de facto para videoxogos de este tamaño.
	\vspace{0.5cm}
\begin{alertblock}{Problemas de simulación}
	Imposibilidade de escalar o tempo sen romper o funcionamento do videoxogo.
\end{alertblock}
\note{
\textbf{Enseñar vídeo}
}
\end{frame}


\subsection{Segunda aplicación}
\begin{frame}
\frametitle{Segunda aplicación}
Implementación de un motor desde cero en C++

\begin{figure}
	\includegraphics[scale=0.3]{../otros/otrasCapturas/block.png} 
\end{figure}

\note{
	HERE
}
\end{frame}


% ========================= EXEMPLOS =========================  %
\section{Sección 1} 
\subsection{Ejemplo de subsección}
\begin{frame}
\frametitle{Título} 
Cada pantalla tiene su título.
\end{frame}

\subsection{Ejemplo de lista}

\begin{frame}
\frametitle{Lista no numerada}
\begin{itemize}
\item una  
\item dos 
\item tres 
\item cuatro
\end{itemize} 
\end{frame}

\begin{frame}
\frametitle{Lista con pausa}
\begin{itemize}
\item número uno \pause 
\item número dos \pause 
\item número tres \pause 
\item número cuatro
\end{itemize} 
\end{frame}

\subsection{Otro ejemplo de lista}
\begin{frame}
\frametitle{Lista numerada}
\begin{enumerate}
\item una  
\item dos 
\item tres 
\item cuatro
\end{enumerate}
\end{frame}

\section{Sección 2} 
\subsection{Tablas}

\begin{frame}
\frametitle{Tablas}
\begin{tabular}{|c|l|r|} \hline
\textbf{Centrado} & \textbf{Izquierda} & \textbf{Derecha} \\ \hline
AAAA  & 1000 & aaaa \\ \hline
BB    & 20   & bb \\ \hline
\end{tabular}
\end{frame}

\begin{frame}
\frametitle{Tabla con pausa}
\begin{tabular}{c c c}
A & B & C \\ \pause 
1 & 2 & 3 \\  \pause 
A & B & C \\ 
\end{tabular} 
\end{frame}

\section{Sección 3}
\subsection{Bloques}

\begin{frame}
\frametitle{Bloques}

\begin{block}{Bloque normal}
Texto del bloque normal
\end{block}

\begin{exampleblock}{Bloque de ejemplo}
Texto del bloque ejemplo
\end{exampleblock}

\begin{alertblock}{Bloque de alerta}
Texto del bloque alerta
\end{alertblock}
\end{frame}

\section{Sección 4}
\subsection{Pantalla dividida}

\begin{frame}
\frametitle{Pantalla dividida}
\begin{columns}
\begin{column}{5cm}
\begin{itemize}
\item una lista
\item de puntos 
\item mas una tabla 
\end{itemize}
\end{column}
\begin{column}{5cm}
\begin{tabular}{|c|c|c|} \hline
\textbf{Mes} & \textbf{Día} & \textbf{Hora} \\ \hline
Enero   & 10 & 15:30 \\ \hline
Febrero & 20 & 20:00 \\ \hline
\end{tabular}
\end{column}
\end{columns}
\end{frame}

\subsection{Figuras} 
\begin{frame}
\frametitle{Incluir figuras}
\begin{figure}
\includegraphics[scale=0.3]{../figuras/logo_usc.eps} 
\caption{Logo de la USC}
\end{figure}
\end{frame}

\subsection{Listas con figuras y pausas} 

\begin{frame}
\frametitle{Listas con figuras y pausas}
\begin{columns}
\begin{column}{4cm}
\begin{itemize}
\item<1-> Una
\item<3-> Dos
\item<5-> Tres
\end{itemize}
\vspace{3cm} 
\end{column}
\begin{column}{4cm}
\begin{overprint}
\includegraphics<2>[scale=0.05]{../figuras/logo_usc.eps}
\includegraphics<4>[scale=0.10]{../figuras/logo_usc.eps}
\includegraphics<6>[scale=0.15]{../figuras/logo_usc.eps}
\end{overprint}
\end{column}
\end{columns}
\end{frame}

\subsection{Cuando se necesita más espacio} 
\begin{frame}[plain]
\frametitle{Pantalla plana con sólo una figura}
\begin{figure}
\includegraphics[scale=0.3]{../figuras/figura01.eps} 
\caption{Una figura grande}
\end{figure}
\end{frame}

\end{document}

