%%% CiTIUS beamer (Template) [Orange]
% 
% Copyright 2017 by Jorge Suárez de Lis
% Copyright 2013 by Diego R. Martinez
%
% This file may be distributed and/or modified
%
% 1. under the LaTeX Project Public License and/or
% 2. under the GNU Public License.


% To use 4:3 instead of 16:9
% 1. Delete "aspectratio=169" from the documentclass options below
% 2. In beamerthemeCiTIUS.sty, uncomment the three lines labeled "% 4:3"
%    and comment the three lines labeled "% 16:9"

\documentclass[11pt,aspectratio=169]{beamer}

\usepackage[english,spanish]{babel}
\usepackage[utf8]{inputenc}
% \usepackage[latin1]{inputenc}
\usepackage[T1]{fontenc}

% Definicións do paquete
%%% CiTIUS beamer (Theme definitions) [Orange]
% 
% Copyright 2017 by Jorge Suárez de Lis
% Copyright 2013 by Diego R. Martinez
%
% This file may be distributed and/or modified
%
% 1. under the LaTeX Project Public License and/or
% 2. under the GNU Public License.


%%% Language, enconding and fonts

%%% Fonts
\usepackage{mathpazo}           % Serif: URW Palladio (~ Palatino)
\usepackage[scaled=.8]{berasans}% Sans serif: Berasans
\usepackage{courier}            % Courier

%%% Style: roman vs. sans serif
% \renewcommand\familydefault{\rmdefault} % SERIF
\renewcommand\familydefault{\sfdefault} % SANS SERIF


%%% CITIUS theme settings
\usetheme[% 
%%% OPTIONS: ==================================
%%%
%%% Guest logos: logos 'invitados'
%           guestlogoone=PATH_TO/logoun,     % primeiro logo
%           guestlogotwo=PATH_TO/logodous,   % segundo logo
%           guestlogothree=PATH_TO/logotres, % terceiro logo
%           guestoffset=0cm,  % desplazamento liña 'invitados' ('+'=esquerda,'-'=dereita)
% Por exemplo:
%           guestlogoone=./figs/logo_invitado,
%%%
%%% Margins 
          marxes=2em,  % tamaño das marxes da área de texto
%%%
%%% Watermark
          watermark=CiTIUS_figs/arbore_vida, % Watermark used in every page.
          watermarkheight=225px,             % Height of the watermark.
%%%
%%% ===========================================
]{CiTIUS}


%%% CITIUS theme switch
\watermarkon % ENABLE watermark
% \watermarkoff % DISABLE watermark (DEFAULT)
%
% \showsubsectionson % ENABLE showsubsection 
% \showsubsectionsoff % DISABLE showsubsection (DEFAULT)
%
% \shownumberpageon % ENABLE shownumber (DEFAULT) 
% \shownumberpageoff % DISABLE shownumber


%%% CITIUS theme logo
\logo{~%
  \pgfuseimage{logo_citius_mini}% BASICO
%   ~~~\includegraphics[height=.038\paperheight]{logo_opcional}%
}

\setbeamercovered{transparent}

%%% COMMANDS
\newcounter{framenumbervorappendix}
\newcommand{\beginbackup}{
  \setcounter{framenumbervorappendix}{\value{framenumber}}
}
\newcommand{\backupend}{
  \addtocounter{framenumbervorappendix}{-\value{framenumber}}
  \addtocounter{framenumber}{\value{framenumbervorappendix}} 
}

\newcommand{\titleslides}{%
  \beginbackup%
  \begin{frame}[plain]%
    \titlepage%
  \end{frame}%
  \backupend%
}

\newcommand{\tocslides}{%
  \beginbackup%
  {%
    \shownumberpageoff% DISABLE page numbering
    \begin{frame}<beamer>{\cabeceiraTOC}%
      \tableofcontents%
    \end{frame}%
  }%
  \backupend%
}

\newenvironment{finishslides}[1]{%
  \section*{}
  \beginbackup%
    \shownumberpageoff% DISABLE page numbering
    \begin{frame}[plain]%
      \bigskip
      \bigskip
      \bigskip
      \bigskip
      \bigskip
      \bigbox{#1}%
      \centering\par
}{%
      \bigskip
      \bigskip
      \bigskip
      \bigskip
      \bigskip
      \bigskip
      \centering\par
      \hbox to \hsize{\hfil{
	\pgfuseimage{logos_feder}
      }\hfil}
      \begin{center}
	{\fontsize{8}{8}\selectfont Cofinanciado pola Unión Europea\\Fondo Europeo de Desenvolvemento Rexional ``Unha maneira de facer Europa``}
      \end{center}
    \end{frame}%
  \backupend%
}

\AtBeginSection[]
{
  \beginbackup%
  {%
    \shownumberpageoff% DISABLE page numbering
    \begin{frame}<beamer>{\cabeceiraTOC}
      \tableofcontents[currentsection]%,currentsubsection]
    \end{frame}
  }%
  \backupend%
}

%%% CORES
\setbeamercolor{orange citius}{fg=white, bg=laranxa citius}
\setbeamercolor{gray citius}{fg=white, bg=gris citius}
\setbeamercolor{grayscale citius}{fg=gris citius oscuro, bg=gris citius claro}
\setbeamercolor{blue citius}{fg=white, bg=azul citius}

%%% COMANDOS
\usepackage{xspace}
\newcommand{\citius}{CiTIUS\xspace}
\newcommand{\CITIUS}{Centro Singular de Investigación en Tecnoloxías da Información\xspace}
\renewcommand{\emph}[1]{\textbf{\color{azul citius}#1}\xspace}

\newlength{\anchopalabro}
\newlength{\altopalabro}

\newcommand{\citiushead}[1]{%
  \settowidth{\anchopalabro}{\ \ \ #1\ \ }%
  \settoheight{\altopalabro}{Tg}%
  \par\vspace{1em}%
  {\color{gris citius}\rule[-.5ex]{.75\textwidth}{1pt}}\hspace*{-.77\textwidth}%
  \begin{beamercolorbox}[wd=\anchopalabro,ht=1.8ex,dp=.5ex]{gray citius}%
  \hspace*{.02\textwidth}#1%
  \end{beamercolorbox}\par%
}

\newcommand{\citiustail}{\par\vspace*{-.25ex}%
  \hspace*{.075\textwidth}{\color{gris citius}%
  \rule[1.5ex]{.95\textwidth}{1pt}\rule[1.5ex]{1pt}{1.5ex}}%
  \hspace*{-.45\textwidth}\par}
  
\newenvironment{caixa}[1]{\citiushead{#1}}{\citiustail}

\newcommand{\bigbox}[1]{%
  \begin{center}\begin{minipage}{.75\textwidth}%
    \begin{beamercolorbox}[sep=1em, rounded=false, center]{orange citius}%
      \Huge #1\vspace*{-1ex}%
    \end{beamercolorbox}%
  \end{minipage}\end{center}%
}


\title{Título da presentación}
\subtitle{Subtítulo da presentación}
\author{Autor(es) da presentación}
\institute{\CITIUS\\ Universidade de Santiago de Compostela}
\date%
  [Título do evento (aparece nos frames)]%
  {Título do evento}

\def\cabeceiraTOC{Cabeceira da tabla de contidos}

  

\begin{document}

\titleslides

\tocslides

%%% Start of demonstration frames.
%%% You can delete from here to the line containing END-DEMO.
%%% Inicio dos frames de mostra.
%%% Podes borrar ata a liña que contén END-DEMO.

\section{Exemplos básicos}
\subsection*{}

\begin{frame}{Frame vacío}{}
\end{frame}


\begin{frame}{Frame con texto}
\framesubtitle{Subtítulo do frame con texto}
 Going to catch the red dot today going to catch the red dot today eat owner's
 food yet poop in litter box, scratch the walls hide at bottom of staircase to
 trip human. Lies down lick butt and make a weird face.\\
 \textit{Texto en cursiva}\\*
 \emph{Texto en negriña}\\*
 \alert{Texto en vermello}
\end{frame}


\begin{frame}{Frame con caixas}
\framesubtitle{As caixas de \texttt{beamer}}

 \begin{block}{Título do block}
   Corpo do block
 \end{block}
 \begin{alertblock}{Título do alertblock}
   Corpo do alertblock
 \end{alertblock}
 \begin{exampleblock}{Título do exampleblock}
   Corpo do exampleblock
 \end{exampleblock}

\end{frame}


\begin{frame}{Frame con caixas}
\framesubtitle{As caixas de \texttt{beamer} (pero sen títulos)}

 \begin{block}{}
   Corpo do block (sen título)
 \end{block}
 \begin{alertblock}{}
   Corpo do alertblock (sen título)
 \end{alertblock}
 \begin{exampleblock}{}
   Corpo do exampleblock (sen título)
 \end{exampleblock}

\end{frame}

\section{Máis funcionalidades}

\subsection*{Listas}

\begin{frame}{Entornos básicos \LaTeX}
\framesubtitle{Un exemplo de \texttt{itemize} con gatos}

\begin{itemize}
 \item litter box
    \begin{itemize}
    \item owner's face
      \begin{itemize}
      \item kitty poochy tuxedo
      \item throwup on your pillow
      \item ball of string
      \end{itemize}
    \item loves cheeseburgers
    \item world domination
    \end{itemize}
 \item chase laser
 \item catch mouse
\end{itemize}

\end{frame}


\begin{frame}{Entornos básicos \LaTeX}
\framesubtitle{Un exemplo de \texttt{description} con gatos}

\begin{description}
 \item[good plan] throwup on your pillow
 \item[better plan] chase laser
 \item[best plan] world domination
\end{description}

\end{frame}

\subsection*{beamercolorbox}

\begin{frame}{Entornos básicos \texttt{beamer}}
\framesubtitle{Un exemplo de \texttt{beamercolorbox}}

\setbeamercolor{blue citius}{fg=white, bg=azul citius}
\setbeamercolor{green citius}{fg=white, bg=verde citius}
\setbeamercolor{orange citius}{fg=white, bg=laranxa citius}
\setbeamercolor{gray citius}{fg=black, bg=gris citius claro}
\setbeamercolor{darkgray citius}{fg=white, bg=gris citius oscuro}

\begin{beamercolorbox}[sep=1ex]{blue citius}
  white \& azul (USC)
\end{beamercolorbox}

\begin{beamercolorbox}[sep=1ex]{green citius}
  white \& verde (Campus Vida)
\end{beamercolorbox}

\begin{beamercolorbox}[sep=1ex]{orange citius}
  white \& orange (CiTIUS)
\end{beamercolorbox}

\begin{beamercolorbox}[sep=1ex]{gray citius}
  black \& gray
\end{beamercolorbox}

\begin{beamercolorbox}[sep=1ex]{darkgray citius}
  white \& darkgray
\end{beamercolorbox}

\end{frame}


\subsection*{beamercolorbox}

\begin{frame}{Entornos básicos \texttt{beamer}}
\framesubtitle{Un exemplo de \texttt{columns}}
  \begin{columns}
  \begin{column}{0.5\textwidth}
    Texto de exemplo.\\
    \begin{itemize}
      \item Meowwww
      \item Run in circles
      \item Crouch, shake butt
    \end{itemize}
  \end{column}
  \begin{column}{0.5\textwidth}
    \begin{beamercolorbox}[sep=1ex]{gray citius}
      Purr while eating peer out window, chatter at birds, lure them to mouth
      for sit in window and stare ooo, a bird!
    \end{beamercolorbox}
    \begin{beamercolorbox}[sep=1ex]{orange citius}
      \textit{being a cat...}
    \end{beamercolorbox}
  \end{column}
\end{columns}
\end{frame}

\section{Aínda máis funcionalidades}

\subsection*{Caixas}

\begin{frame}{Entornos propios desta plantilla}
\framesubtitle{Un exemplo de \texttt{caixa}}
 \begin{caixa}{Título da caixa}
  Contido da caixa.\\
  Poden ser varias \textit{liñas}.
 \end{caixa}
\end{frame}

\subsection*{Bigbox}

\begin{frame}{Entornos propios desta plantilla}
\framesubtitle{Un exemplo de \texttt{bigbox}}
 \bigbox{Algo destacado moi en grande!}
\end{frame}


%%% END-DEMO

\begin{finishslides}{Mensaxe final}
  Texto adxunto á mensaxe final
\end{finishslides}

\end{document}
